% CV LaTeX Template pour Manoel Da Ponte
% Basé sur le style moderncv

\documentclass[11pt,a4paper,sans]{moderncv}

% Configuration des couleurs - Bleu similaire à votre CV actuel
\definecolor{mainblue}{RGB}{40, 75, 129}
\definecolor{lightblue}{RGB}{110, 180, 215}
\colorlet{awesome}{mainblue}
\moderncvstyle{classic}
\moderncvcolor{blue}

% Ajustements pour la page
\usepackage[scale=0.82]{geometry}
\setlength{\hintscolumnwidth}{3.5cm}

% Informations personnelles
\name{Manoel}{Da Ponte}
\title{Ingénieur Data Scientist}
\phone[mobile]{+33 6 82 69 71 32}
\email{daponte.manoel@gmail.com}
\address{Avenue Emmanuel Maignan}{}{Toulouse}
\extrainfo{Permis B + véhicule}

\begin{document}

\makecvtitle

% Profil personnel
\section{Profil Personnel}
Spécialisé dans le traitement de données (image, texte, son, numérique) de leur collection à leur utilisation dans la prise de décision, avec plus de deux ans d'expérience professionnelle, je suis à la recherche d'opportunités dans le domaine de la data science.

% Savoir-être et savoir-faire
\section{Savoir-être et Savoir-faire}
Personne motivée et disciplinée - Trouve de la joie à aider les autres et de l'inspiration à être aidé - Connaissances professionnelles en anglais - Habitué aux méthodes de gestions de projets SCRUM and SAFe - Complété par des notions de marketing et d'économie.

% Compétences
\section{Compétences}
\begin{itemize}
\item \textbf{Python:} Numpy, Pandas, Keras, Scikit Learn, NLTK...
\item \textbf{R:} Rstudio, PCA, Anova...
\item \textbf{Data Viz:} Qlik Sense, Power BI
\item \textbf{BDD:} SQL / PL SQL, NoSQL (MongoDB, Neo4j)
\item \textbf{ETL:} Talend, OWB
\item \textbf{Concepts:} Machine Learning, Deep Learning, Data Warehouse, Analyse de données, Big Data, Scrapping...
\item \textbf{Dev Ops:} Git, Bash, Jenkins, Docker
\item \textbf{Autres:} Jira, Excel, Java
\end{itemize}

% Expériences Professionnelles
\section{Expériences Professionnelles}

\cventry{02/2021--Présent}{Alternant Data Scientist}{CGI}{}{}{
Au sein d'une équipe projet de 12 personnes créant une application d'aide à la gestion des ressources humaines. Création et intégration d'un Chat Bot ayant pour but d'aider à la navigation, et d'échanger avec l'utilisateur.
\newline\textit{Technologies: Python, NLP, SQL Server, Docker, Azure}}

\cventry{06/2020--09/2020}{Chargé de Travaux Pratiques}{Université Paul Sabatier}{}{}{
Création d'exercices pratiques dans le but de mettre en pratique les connaissances théoriques des étudiant en master de mathématique.
\newline\textit{Technologies: Python, R, Mathématiques, Pédagogie}}

\cventry{05/2019--08/2019}{Stagiaire Business Intelligence}{Capgemini}{}{}{
Suivi et déploiement d'un outil permettant la centralisation de données provenant de projets internes. Tests de non-régression, Data Viz. Création d'un algorithme de détection de projets risqués et prédictions de projets à perte.
\newline\textit{Technologies: Qlik Sense, BI, Tests, Python, Classification}}

\cventry{05/2018--08/2018}{Stagiaire Machine Learning}{Sopra Stéria}{}{}{
Création et déploiement d'un algorithme basé sur des méthodes de Machine Learning. Détection d'anomalies et prédiction des sources d'anomalies sur les données provenant des logiciels embarqués d'avion Airbus. Automatisation de l'apprentissage, analyse et suivi de l'outil.
\newline\textit{Technologies: Python, Docker, Classification, NLP}}

% Diplômes et Formations
\section{Diplômes et Formations}

\cventry{09/2018--09/2021}{Master Statistiques et Informatique Décisionnelle}{Université Paul Sabatier}{Toulouse}{En cours}{}

\cventry{09/2017--09/2018}{Licence 3 Mathématiques et Informatiques Appliquées aux Sciences Humaines}{Université Paul Sabatier}{Toulouse}{Obtenu}{}

\cventry{09/2015--09/2017}{DUT Statistiques et Informatique Décisionnelle}{}{}{Obtenu}{}

% Projets / Challenges
\section{Projets / Challenges}

\cventry{10/2020--03/2021}{Compétition Kaggle - Classification / Fairness}{}{}{}{
En compétition avec plusieurs écoles internationales. Classification de métier en fonction de descriptions textuelles sur la personne. Double évaluation : performance de classification / correction du biais engendré par le genre.
\newline\textit{Technologies: Python, NLP, Classification, Éthique, Compétition}}

\cventry{10/2019--03/2020}{Compétition Kaggle / Airbus - Détection d'Anomalies}{}{}{}{
En collaboration avec Airbus, challenge visant à détecter de potentielles anomalies en se basant sur un grand nombre de données provenant de logiciels embarqués lors des vols.
\newline\textit{Technologies: Python, Machine Learning, Compétition}}

\cventry{01/2019--02/2019}{Projet Équipe Airbus - Paul Sabatier - Traitements d'Images}{}{}{}{
Reconnaissance des modèles d'avions depuis des photos prises soit de l'intérieur soit de l'extérieur de l'appareil. Utilisation de méthodes de Transfert Learning.
\newline\textit{Technologies: Python, Image, Deep Learning}}

\end{document}